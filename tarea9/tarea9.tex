\documentclass{article}

\usepackage[utf8x]{inputenc}
\usepackage[spanish]{babel}
\usepackage[margin=3.1cm]{geometry}
\usepackage{amsmath}
\usepackage{amssymb}
\usepackage{graphicx}
\usepackage{algorithm}
\usepackage{algorithmic}

\usepackage{ stmaryrd }

\usepackage{ upgreek }

\usepackage{listings}

\linespread{1.2}

\usepackage{color}
\usepackage{listings}

\definecolor{javared}{rgb}{0.6,0,0} % for strings
\definecolor{javagreen}{rgb}{0.25,0.5,0.35} % comments
\definecolor{javapurple}{rgb}{0.5,0,0.35} % keywords
\definecolor{javadocblue}{rgb}{0.25,0.35,0.75} % javadoc
 
\lstset{language=Java,
basicstyle=\ttfamily,
keywordstyle=\color{javapurple}\bfseries,
stringstyle=\color{javared},
commentstyle=\color{javagreen},
morecomment=[s][\color{javadocblue}]{/**}{*/},
numbers=left,
numberstyle=\tiny\color{black},
stepnumber=1,
numbersep=10pt,
tabsize=4,
showspaces=false,
showstringspaces=false}

\title{ Computación Concurrente \\ \Large{Tarea 9}
\author{
  Diego Goméz Montesinos
  \and
  José Emiliano Cabrera Blancas
  }
\date{6 Mayo 2014}
}
\begin{document}
\maketitle
\begin{enumerate}
  
\item{
    \textsl{
      Los programadores de la compañía de computación Flaky diseñaron
      el protocolo que se muestra a continuación para lograr la
      exclusión mútua de n-hilos. Para cada pregunta, presenta una
      prueba o despliega una ejecución donde falle.
    }
      \begin{itemize}
        \item{\textsl{¿El protocolo satisface la exclusión mútua?}}
        \item{\textsl{¿Este protocolo es starvation-free?}}
        \item{\textsl{¿Este protocolo es deadlock-free?}}
      \end{itemize}

      \renewcommand{\lstlistingname}{}
\begin{lstlisting}[frame=single]
class Flaky implements Lock {
   private volatile int turn ;
   private volatile boolean busy = false ; 
   
   public void lock () {
      int me = ThreadID.get();
      do {
         do {
            turn = me;
         } while (busy);
         busy = true;
         } while (turn != me);
   }

   public void unlock () {
      busy = false;
   }
}
\end{lstlisting}
      
  }

\item{
    \textsl{
      Durante la exposición se mostró la construcción de un sistema
      bounded-Timestamp para 2 y 3 hilos, con base en la explicación
      dada, construya uno para 4 hilos y responda adicionalmente las
      siguientes preguntas.
    }
    \begin{itemize}
    \item{\textsl{
          En el sistema construido para 3 hilos puede observarse que
          existen más vértices que procesos y podría suponerse que es
          posible el utilizarlo para 4 hilos, muestre un ejemplo que
          contradiga esta suposición.
        }}

    \item{\textsl{
          Indique el número de vértices necesarios para un sistema de
          8 hilos e indique el número máximo de bits necesarios para
          representarlo.
        }}
    \end{itemize}
  }
    
\item{
    \textsl{
      Lee el documento de Moore´s Law and the Sand-Heap Paradox y da
      tu opinión en una cuartilla.
    }
  }
  
\end{enumerate}
\end{document}
