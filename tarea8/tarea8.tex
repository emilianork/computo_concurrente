\documentclass{article}

\usepackage[utf8x]{inputenc}
\usepackage[spanish]{babel}
\usepackage[margin=3.1cm]{geometry}
\usepackage{amsmath}
\usepackage{amssymb}
\usepackage{graphicx}
\usepackage{algorithm}
\usepackage{algorithmic}

\usepackage{ stmaryrd }

\usepackage{ upgreek }

\usepackage{listings}

\linespread{1.2}

\title{ Computación Concurrente \\ \Large{Tarea 8}
\author{
  Diego Goméz Montesinos
  \and
  José Emiliano Cabrera Blancas
  }
\date{22 Abril 2014}
}
\begin{document}
\maketitle
\begin{enumerate}
  
\item{
    \textsl{
      En la fábula entre productores y consumidores, se asumió que Bob
      puede ver si la lata en la ventana de Alice esta arriba o
      abajo. Diseña un protocolo utilizando latas y cuerdas que
      funciona si Bob no puede ver el estado de las latas. Demuestra
      que el protocolo cumple con safety y liveness.
    }
  }
  
\item{
    \textsl{
      Suponiendo que se tiene un programa 30\% serial y 70\%
      paralelizable. Si queremos hacerlo n veces más rápido, cuantos
      núcleos de proceso serán necesarios.
    }
  }

\end{enumerate}
\end{document}
