\documentclass{article}

\usepackage[utf8x]{inputenc}
\usepackage[spanish]{babel}
\usepackage[margin=3.1cm]{geometry}
\usepackage{amsmath}
\usepackage{amssymb}
\usepackage{graphicx}
\usepackage{algorithm}
\usepackage{algorithmic}

\usepackage{ stmaryrd }

\usepackage{ upgreek }

\usepackage{listings}

\linespread{1.2}

\title{ Computación Concurrente \\ \Large{Tarea 8}
\author{
  Diego Goméz Montesinos
  \and
  José Emiliano Cabrera Blancas
  }
\date{22 Abril 2014}
}
\begin{document}
\maketitle
\begin{enumerate}
  
\item{
    \textsl{
      En la fábula entre productores y consumidores, se asumió que Bob
      puede ver si la lata en la ventana de Alice esta arriba o
      abajo. Diseña un protocolo utilizando latas y cuerdas que
      funciona si Bob no puede ver el estado de las latas. Demuestra
      que el protocolo cumple con safety y liveness.
    }
    
    Los algoritmos para establecer el protocolo son los siguientes:
    \begin{itemize}
      \item{
          Alice(consumidor):
          \begin{itemize}
            \item[1.]{Espera a que la lata este arriba.}
            \item[2.]{Saca al perro.}
            \item[3.]{Espera a que el perro consuma la comida.}
            \item[4.]{Mete al perro.}
            \item[5.]{Tira la lata.}
          \end{itemize}
        }

      \item{
          Bob(productor):
          \begin{itemize}
            \item[1.]{Espera  a que la lata este abajo.}
            \item[2.]{Sale a dejar comida.}
            \item[3.]{Regresa}
            \item[4.]{Coloca arriba la lata.}
          \end{itemize}
        }
    \end{itemize}

    Demostaremos que nuestro protocolo cumple con las siguientes
    propiedades:

    \begin{itemize}
      \item{\textit{Safety (mutual exclusion)}: Bob y los perros no
          pueden estar en el jardín al mismo tiempo.}
      \item{\textit{Liveness (starvation-freedom) }: Si Bob siempre
          tiene comida, y los perros siempre tienen hambre, entonces
          los perros comeran de forma indefinidamente.}

        \item{\textit{Producer-Consumer}:Los perros no saldrán al jardín a
        menos de que alla comida ahí,  y Bob no proveera de comida a
        menos de que la comida se alla consumido. }
    \end{itemize}
    
    Desmostración (\textit{Safety}):\\
    
    La lata que Alice tira y Bob levanta se puede representar con un
    \textit{bit}, en donde si el \textit{bit} esta prendido,
    representa a la lata parada, y si el \textit{bit} esta apagado,
    representa a la lata tirada. Para esta demostración utilizaremos
    la memoria compartida que Bob prende y Alice apaga.\\
    Supongamos que se da el caso donde Alice saca a los perros y Bob
    esta en el jardín, eso quiere decir que Bob vio el \textit{bit}
    apagado y Alice vio el \textit{bit} prendido. Para que Bob vea el
    \textit{bit} apagado, quire decir que Alice lo apago, puesto que
    es el único pedaso del protocolo donde el \textit{bit} se
    apaga, pero para que eso ocurra Alice antes debio meter a los
    perros, lo cual es una contracción, puesto que los perros estan
    afuera en el jardín y la única posibilidad de que esten afuera es
    que Bob haya prendido el \textit{bit} justo cuando regresa a su
    casa.\\
    Por lo tanto la propiedad de \textit{safety} se cumple.\\
    
    Demostración (\textit{Liveness}):\\
    
    Supongamos que los perros siempre tienen hambre y Bob intenta
    salir a dejar comida y no puede. Eso quiere decir que el
    \textit{bit} no puede estar apagada por que Bob intentaria salir a
    dejar comida, pero entonces el \textit{bit} esta prendido, pero
    como los perros tienen hambre, eventualmente Alice apagara el
    \textit{bit}, por lo tanto debe de estar apagado, lo cual es una
    contradicción. Por lo tanto la propiedad de
    \textit{starvetion-freedom} se cumple.\\

    Demostración (\textit{Producer-Consumer}):\\

    La propiedad de \textit{Safety} nos asegura que los perros y Bob
    nunca estaran juntos en el jardin. Bob no entrar al jardín hasta
    que Alice apague el \textit{bit}, que solo hará si no hay mas
    comida en el jardón. De forma similar, los perros no entran al
    jardín hasta que Bob prenda el \textit{bit}, que solo hara después
    de colocar comida en el jardín. Por lo tanto los perros salen
    cuando no hay comida y Bob no proveera de comida a menos de que
    previamente los perros se la hayan comido.\\
  }
  
\item{
    \textsl{
      Suponiendo que se tiene un programa 30\% serial y 70\%
      paralelizable. Si queremos hacerlo n veces más rápido, cuantos
      núcleos de proceso serán necesarios.
    }
  }

\end{enumerate}
\end{document}
