\documentclass{article}

\usepackage[utf8x]{inputenc}
\usepackage[spanish]{babel}
\usepackage[margin=3cm]{geometry}
\usepackage{amsmath}
\usepackage{amssymb}
\usepackage{graphicx}

\linespread{1.2}

\title{Computación Concurrente \\ \Large{Tarea Exámen 1}}
\author{
  Diego Goméz Montesinos
  \and
  José Emiliano Cabrera Blancas
  }
\date{4 marzo 2014}
\begin{document}
\maketitle
\begin{enumerate}
  
\item{
    \textsl{
      En clase hemos visto mapas portadores y mapas simpliciales. Sea $\Xi$ un
      mapa portardor y sea $\delta$ un mapa simplicial que preserva estructura,
      demuestra que si la composición $\delta \circ \Xi$ es un mapa portador, 
      entonces $\Xi \circ \delta$ es un mapa portador.\\
    }
    
    Por demostrar:\\
    Si la composición $\delta \circ \Xi$ es un mapeo portador $\Rightarrow$ 
    $\Xi \circ \delta$ es un mapeo portador.
    
    Demostración:\\
    Primero hacemos una observación, sea $v \in G$, por hipótesis la siguiente 
    proposición se cumple: $\delta(v) \in G$ y $\Xi(v) \in G$.\\
    Dado que $\Xi$ es un mapeo portador, sabemos que si $\tau \in G$ y 
    $\sigma \in G$ $\Rightarrow$ $\Xi(\tau)$ $\subseteq$ $\Xi(\sigma)$(es monótono).\\
    Por lo que si $\delta(\Xi(G))$ es un mapeo portador, entonces $\delta(u)$ es un
    mapeo portador con $u \in \Xi(G)$, dicho de otra forma, $\tau' \in \Xi(G)$ y 
    $\sigma' \in \Xi(G)$, donde $\tau'$ $\subseteq$ $\sigma'$ $\Rightarrow$ $\delta(\tau') 
    \subseteq \delta(\sigma')$.\\
    Como $\Xi(G) = G$, entonces por hipótesis $\delta$ es un mapeo portador, por lo que 
    
  }
  
\item{
    \textsl{ 
      Diseña un algoritmo (secuencial) que dada una tarea $T = (\mathcal{I},\mathcal{O},\Delta)$
      conteste si tiene o no solución en el modelo iterado wait-free visto en clase, y si la tiene,
      conteste en a lo más cuantas capas. Analiza su correctez y complejidad (como función del tamaño
      de la entrada), suponiendo que para cada vértice $v$ de $\mathcal{I},\Delta(v)$ consiste de 
      un conjunto de a lo más $k$ vértices, para una constante k.
    }
  }    

\item {
    \textsl{
      En el modelo anónimo iterado para n procesos, $n ≥ 1$, con $L = 1$ iteración, definir
      cuales son las posibles vistas de los procesos en una ejecución, cuyos valores de entrada
      son $S$. Es decir, para cada $x \in S$, al menos un proceso empieza con $x$. (posta: si el
      conjunto de entradas en la ejecución es un conjunto $S$, la vista de un proceso es un 
      subconjunto de $S$, y las vistas $S_1,S_2,...,S_k$ están en la misma ejecución si y solo si
      todas son subconjuntos de $S$, y se pueden ordenar de forma que cada una este contenida o
      sea igual a la siguiente)
    }
  }
  
\item{
    \textsl{
      Demuestra que un modelo anónimo y un modelo cromático tienen el mismo poder de cómputo, es
      decir, todas las tareas anónimas, $\langle \mathcal{I},\mathcal{O},\Delta \rangle$ tal que
      $\mathcal{I}$ y $\mathcal{O}$ no tiene colores, que se pueden resolver en un modelo anónimo
      también las puede resolver un modelo cromático.\\
      Considera modelos para dos procesos, iterado y wait-free.\\
    }

    Demostración:\\
    Sea $T$ una tarea definida por $\langle$ $\mathcal{I},\mathcal{O},\Delta$ $\rangle$ con $\mathcal{I}$
    y $\mathcal{O}$ gráficas anónimas. Por hipótesis decimos que existe una $\delta$ un mapeo portador
    tal que resuelve la tarea dada.
    ¿Qué un modelo sea anónimo que quiere decir?, visto de otra forma, podemos decir que la $\delta$ que
    resuelve el modelo anónimo tiene la caracteristica de no importarle que vértice es, sino solo la 
    información que el vértice(proceso) escribio en memoria (la $\delta$ es una función de equidad de género).\\
    Sea $T'$ una tarea definida por $\langle$ $\mathcal{I}',\mathcal{O}',\Delta'$ $\rangle$ con $\mathcal{I}'$
    y $\mathcal{O}'$ gráficas cromáticas, tal que la tarea $T'$ es la versión del problema $T$ en el modelo 
    cromático.\\
    ¿Siempre existe una $\delta$ que resuelva esta $T'$?, podemos construir una $\delta'$ que resuelve la tarea $T'$
    a partir de la $\delta$, simplemente le quitamos los colores a los vértices y se los pasamos a la función $\delta$.
    Por lo tanto $\delta'$ resuelve la tarea $T'$.\\
    Por lo tanto el modelo cromático tiene el mismo poder de computo que el modelo anónimo.
    
  }
\end{enumerate}
\end{document}