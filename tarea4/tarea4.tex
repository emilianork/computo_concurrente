\documentclass{article}

\usepackage[utf8x]{inputenc}
\usepackage[spanish]{babel}
\usepackage[margin=3cm]{geometry}
\usepackage{amsmath}
\usepackage{amssymb}
\usepackage{graphicx}
\usepackage{algorithm}
\usepackage{algorithmic}

\linespread{1.2}

\title{Computación Concurrente \\ \Large{Tarea Exámen 1}}
\author{
  Diego Goméz Montesinos
  \and
  José Emiliano Cabrera Blancas
  }
\date{4 marzo 2014}
\begin{document}
\maketitle
\begin{enumerate}
  
\item{
    \textsl{
      En clase hemos visto mapas portadores y mapas simpliciales. Sea $\Xi$ un
      mapa portardor y sea $\delta$ un mapa simplicial que preserva estructura,
      demuestra que si la composición $\delta \circ \Xi$ es un mapa portador, 
      entonces $\Xi \circ \delta$ es un mapa portador
    }
  }
  
\item{
    \textsl{ 
      Diseña un algoritmo (secuencial) que dada una tarea $T = (\mathcal{I},\mathcal{O},\Delta)$
      conteste si tiene o no solución en el modelo iterado wait-free visto en clase, y si la tiene,
      conteste en a lo más cuantas capas. Analiza su correctez y complejidad (como función del tamaño
      de la entrada), suponiendo que para cada vértice $v$ de $\mathcal{I},\Delta(v)$ consiste de 
      un conjunto de a lo más $k$ vértices, para una constante k.
    }

    \textbf{SOLUCIÓN}\\
    Primero veamos el algoritmo verificador.\\
    En clase vimos el Teorema Principal para tareas de dos procesos: Dada una tarea $(\mathcal{I},\mathcal{O},\Delta)$
    tiene solución en el modelo iterado por capas si y solo si existe un mapeo portador conexo
    $\Phi: \mathcal{I} \rightarrow 2^{\mathcal{O}}$, portado por $\Delta$.\\
    Dado que el modelo iterado wait-free es un subconjunto del modelo iterado por capas, entonces, podemos
    aplicar el Teorema y decir que la tarea $T$ tiene solución si existe un mapeo portador conexo $\Phi$, portado por $\Delta$.\\
    Entonces el algoritmo tendría que verificar si existe o no tal mapeo portador conexo; ahora, dado que el mapeo portador
    $\Phi$ debe ser conexo, entonces debe ocurrir que $\forall\sigma\in\mathcal{I}$, $\Phi(\sigma)$ es una
    gráfica conexa. Pero como $\Phi$ es portado por $\Delta$, entonces tendríamos que ver que $\Delta$ fuera un mapeo portador
    conexo, ya que de lo contrario $\Phi$ no sería conexo.\\
    Entonces, la idea del algoritmo será recorrer $I$, e ir checando que $\Delta$, sea un mapeo portador conexo.
    Además llevaremos un contador de aristas, para ver cual es el camino más largo y en base a eso decir cuantas capas
    son necesarios para resolverlo.\\
    Dada la definición de mapeo portador conexo, el algoritmo quedaría de la siguiente manera:\\
    
    \begin{algorithmic}
    \STATE $m \leftarrow 0$
    \STATE $k \leftarrow 0$
    \FORALL{arista $e\in\mathcal{I}$ \COMMENT{Esto se puede hacer con BFS ó DFS, empezando con una arista arbitraria}}
    \STATE $e \leftarrow $ arista actual
    \STATE $u \leftarrow $ un extremo de $e$
    \STATE $v \leftarrow $ el otro extremo de $e$
    \STATE $esConexo \leftarrow False$
    \FORALL{arista $e'\in\Delta(e)$ \COMMENT{Esto se puede hacer con BFS ó DFS, empezando con una arista adyacente a $\Delta(u)$}}
    \STATE $e' \leftarrow $ arista actual
    \IF{$e' \in \Delta(e)$}
    \STATE $u' \leftarrow $ un extremo de $e'$
    \STATE $v' \leftarrow $ el otro extremo de $e'$
    \IF{$u' == \Delta(v) \wedge v' == \Delta(v) $}
    \STATE $esConexo \leftarrow True$
    \ENDIF
    \STATE $k \leftarrow k + 1$
    \ENDIF
    \ENDFOR
    \IF{$!esConexo$}
    \RETURN $False$
    \ENDIF
    \STATE $m \leftarrow max(k, m)$
    \ENDFOR
    \RETURN $\lceil log_{3} m \rceil$
    \end{algorithmic}

    Dado que DFS y BFS son de complejidad $O(\mathcal{I}) = O(|V| + |E|)$, y suponemos que cada vértice $v$ de $\mathcal{I},\Delta(v)$
    consiste de un conjunto de a lo más $k$ vértices, para una constante k; entonces la complejidad del algoritmo es de
    $O((|V| + |E|) k)$.\\
    El algoritmo es correcto, ya que tanto $\mathcal{I}$, como $\mathcal{O}$ son finitas, es decir, ambos recorridos terminarán; además
    cumplirá con descubrir si algún $\Delta(e)$ no es conexo, y con esto ver si $\Delta$ es un mapeo portador conexo. Note que, por
    hacer el recorrido BFS o DFS nos vamos tomando aristas adyacentes, verificando de igual manera que si $v\subseteq\sigma\cap\tau$ entonces
    $\Delta(v)\subseteq\Delta(\sigma)\cap\Delta(\tau)$.
  }

\item {
    \textsl{
      En el modelo anónimo iterado para n procesos, $n ≥ 1$, con $L = 1$ iteración, definir
      cuales son las posibles vistas de los procesos en una ejecución, cuyos valores de entrada
      son $S$. Es decir, para cada $x \in S$, al menos un proceso empieza con $x$. (posta: si el
      conjunto de entradas en la ejecución es un conjunto $S$, la vista de un proceso es un 
      subconjunto de $S$, y las vistas $S_1,S_2,...,S_k$ están en la misma ejecución si y solo si
      todas son subconjuntos de $S$, y se pueden ordenar de forma que cada una este contenida o
      sea igual a la siguiente)
    }
  }
  
\item{
    \textsl{
      Demuestra que un modelo anónimo y un modelo cromático tienen el mismo poder de cómputo, es
      decir, todas las tareas anónumas, $\langle \mathcal{I},\mathcal{O},\Delta \rangle$ tal que
      $\mathcal{I}$ y $\mathcal{O}$ no tiene colores, que se pueden resolver en un modelo anónimo
      también las puede resolver un modelo cromático.\\
      Considera modelos para dos procesos, iterado y wait-free.
    }
  }
\end{enumerate}
\end{document}