\documentclass{article}

\usepackage[utf8x]{inputenc}
\usepackage[spanish]{babel}
\usepackage[margin=3cm]{geometry}
\usepackage{amsmath}
\usepackage{amssymb}
\usepackage{graphicx}
\usepackage{algorithm}
\usepackage{algorithmic}

\usepackage{ upgreek }

\usepackage{listings}

\linespread{1.2}

\title{ Computación Concurrente \\ \Large{Tarea 6}
\author{
  Diego Goméz Montesinos
  \and
  José Emiliano Cabrera Blancas
  }
\date{20 marzo 2014}
}
\begin{document}
\maketitle
\begin{enumerate}
  
\item{
    \textsl{
      Leer y escribir un resumen de no más de 3 páginas: Given Tables
      but No Teachers, Ethiopian Children Teach Themselves, en MIT
      Technology Review, by David Talbot on October 29, 2012||
      http://www.technologyreview.com/news/506466/given-tables-but-not-teachers-ethiopian-children-teach-themselves.\\
    }
    
  }
  
  \item{
      \textsl{
    Sea $\mathcal{S}$ un conjunto de simplejos de $\mathcal{A}$ un
    complejo de $\mathcal{S}$ $\in$ $\mathcal{A}$. Investiga las
    definiciones formales, describe cada una con tus propias palabras
    e incluye las referencias consultadas:
    }
    \begin{itemize}
    \item{Stº($\mathcal{S}$, $\mathcal{A}$)}
      
    \item{St($\mathcal{S}$, $\mathcal{A}$)}

    \item{Lk($\mathcal{S}$, $\mathcal{A}$)}

    \item{Dl($\mathcal{S}$,$\mathcal{A}$)}
    \end{itemize}

  
  }

\item{
    \textsl{
      Realiza las operaciones y muestra de manera geométrica el
      complejo inicial y el resultado después de aplicar la operación,
      además indica las dimensiones de los simplejos contenidos en el
      complejo resultante.
    }

    \begin{itemize}
    \item{
        Sea $\mathcal{K}$ $=$ $\{\{\}, \{0\}, \{1\}, \{2\}, \{3\},
        \{0,1\}, \{0,2\}, \{1,2\}, \{1,3\}, \{2,3\}, \{0,1,2\},
        \{1,2,3\} \}$ y sea $\tau$ $=$ $\{3\}$. $Dl(\tau,\mathcal{K})$
      }
      
    \item{

        Sea $\mathcal{K}$ $=$ $\{\{\}, \{0\}, \{1\}, \{2\}, \{3\},
        \{0,1\}, \{0,2\}, \{0,3\}, \{1,2\}, \{1,3\}, \{2,3\}, \{0,1,3\},
        \{0,2,3\}, \{1,2,3\}\ \}$ y sea $\tau$ $=$ $\{3\}$. $Lk(\tau,\mathcal{K})$
        
      }

    \item{
      }

    \item{
      }
    \end{itemize}

    
  }

\end{enumerate}
\end{document}