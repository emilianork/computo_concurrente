\documentclass{article}

\usepackage[utf8x]{inputenc}
\usepackage[spanish]{babel}
\usepackage[margin=3.1cm]{geometry}
\usepackage{amsmath}
\usepackage{amssymb}
\usepackage{graphicx}
\usepackage{algorithm}
\usepackage{algorithmic}

\usepackage{ upgreek }

\usepackage{listings}

\linespread{1.2}

\title{ Computación Concurrente \\ \Large{Tarea 6}
\author{
  Diego Goméz Montesinos
  \and
  José Emiliano Cabrera Blancas
  }
\date{20 marzo 2014}
}
\begin{document}
\maketitle
\begin{enumerate}
  
\item{
    \textsl{
      Leer y escribir un resumen de no más de 3 páginas: Given Tables
      but No Teachers, Ethiopian Children Teach Themselves, en MIT
      Technology Review, by David Talbot on October 29, 2012\\
      http://www.technologyreview.com/news/506466/given-tables-but-not-teachers-ethiopian-children-teach-themselves.\\
    }
    
  }
  
  \item{
      \textsl{
    Sea $\mathcal{S}$ un conjunto de simplejos en $\mathcal{A}$ un
    complejo tal que $\mathcal{S}$ $\in$ $\mathcal{A}$. Investiga las
    definiciones formales, describe cada una con tus propias palabras
    e incluye las referencias consultadas:
    }
    \begin{itemize}
      
    \item{
          \[
           Stº(\mathcal{S}, \mathcal{A}) =  \bigcup_{\mathcal{S}
             \subseteq \tau} Int \tau 
          \]

          Donde $Int\tau$ es el interior de los simplejos $\tau$.\\\\
          Podemos pensar que la operación $open$ $star$ de un simplejo
          $\mathcal{S}$ es la union de
          todos los simplejos $\tau$ que contienen a $\mathcal{S}$
          tal que le quitamos la frontera a los simplejos $\tau$
          (como si fueran intervalos abiertos).
      }
      
    \item{
        \[
           St(\mathcal{S}, \mathcal{A}) =  \bigcup_{\mathcal{S}
             \subseteq \tau} \tau 
           \]
           Intuitivamente podemos ver que la operación $star$ de un
           simplejo $\mathcal{S}$ es igual que la operación $open$
           $star$ pero sin quitarle la frontera a los simplejos
           $\tau$(como si fueran intervalos cerrados).
         }

    \item{
        \[
        Lk(\mathcal{S}, \mathcal{A})
        \]
        Es el subcomplejo de $\mathcal{A}$ que consiste de todos los
        simplejos en $St(\mathcal{S},\mathcal{A})$ que no tienen
        vertices en comun con $\mathcal{S}$,
      }

    \item{
        \[
        Dl(\mathcal{S},\mathcal{A})
        \]
        La operación $deletion$ de $\mathcal{S}$ $\in$ $\mathcal{A}$
        es el sucomplejo de $\mathcal{A}$ que consiste de todos los
        simplejos que no tienen vertices en común con $\mathcal{S}$.
        }
    \end{itemize}

  
  }

\item{
    \textsl{
      Realiza las operaciones y muestra de manera geométrica el
      complejo inicial y el resultado después de aplicar la operación,
      además indica las dimensiones de los simplejos contenidos en el
      complejo resultante.
    }

    \begin{itemize}
    \item{
        \textsl{
        Sea $\mathcal{K}$ $=$ $\{\{\}, \{0\}, \{1\}, \{2\}, \{3\},
        \{0,1\}, \{0,2\}, \{1,2\}, \{1,3\}, \{2,3\}, \{0,1,2\},
        \{1,2,3\} \}$ y sea $\tau$ $=$ $\{0,1\}$. $Dl(\tau,\mathcal{K})$
      }
      }
      
    \item{
        \textsl{
        Sea $\mathcal{K}$ $=$ $\{\{\}, \{0\}, \{1\}, \{2\}, \{3\},
        \{0,1\}, \{0,2\}, \{0,3\}, \{1,2\}, \{1,3\}, \{2,3\}, \{0,1,3\},
        \{0,2,3\}, \{1,2,3\}\}$ y sea $\tau$ $=$ $\{3\}$. $Lk(\tau,\mathcal{K})$
      }
    }

    \item{
        \textsl{
          Sea $\mathcal{K}$ $=$ $\{ \{ \}, \{0\} , \{1\}, \{2\}, \{3\}, \{0, 1\},\{0,
              2\}, \{0, 3\},\{1, 2\}, \{1, 3\}, \{2, 3\},
              \{0,1,3\},\{0,2,3\},\{1,2,3\}\}$ y sea $\tau$ $=$ $\{0\}$. $St(\tau,\mathcal{K})$
        }
      }

    \item{
        \textsl{
          Sea $\mathcal{K}_1$ $=$
          $\{\{\},\{0\},\{1\},\{2\},\{0,1\},\{0,2\},\{1,2\},\{0,1,2\}\}$
          y sea $\mathcal{K}_2$ $=$ $\{\{3\}, \{4\}, \{3,
          4\}\}$. $\mathcal{K}_1$ $*$ $\mathcal{K}_2$
        }
      }
    \end{itemize}
  }

  \item{
      \textsl{
        Demuestra que un modelo cromático tiene el mismo poder de
        cómputo que un modelo anónimo, es decir, todas las tareas
        anónimas, $\langle\mathcal{I},\mathcal{O},\Delta\rangle$ tal
        que $\mathcal{I}$ y $\mathcal{O}$ no tiene colores, que se
        pueden resolver en un modelo anónimo también las puede
        resolver un modelo cromático y viceversa.\\
        Considera modelos para tres procesos, iterado y wait-free.\\
      }
    }

  \item{
    \textsl{
      Define y responde una pregunta que te gustaría realizar como
      tarea. (Consideraciones para evaluar: Calidad del tema,
      argumentos y claridad en la justificación de la respuesta,
      conocimientos generados y/o reforzados) 3.5 pts.\\
    }
  }

\end{enumerate}
\end{document}