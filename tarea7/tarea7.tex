\documentclass{article}

\usepackage[utf8x]{inputenc}
\usepackage[spanish]{babel}
\usepackage[margin=3.1cm]{geometry}
\usepackage{amsmath}
\usepackage{amssymb}
\usepackage{graphicx}
\usepackage{algorithm}
\usepackage{algorithmic}

\usepackage{ stmaryrd }

\usepackage{ upgreek }

\usepackage{listings}

\linespread{1.2}

\title{ Computación Concurrente \\ \Large{Tarea 7}
\author{
  Diego Goméz Montesinos
  \and
  José Emiliano Cabrera Blancas
  }
\date{1 Abril 2014}
}
\begin{document}
\maketitle
\begin{enumerate}
  
  \item{
    \textsl{
      Considera el modelo sincrono de envio de mensajes para tres
      procesos, en el que a lo más 2 procesos pueden fallar (en
      cualquier ejecución del sistema, pueden fallar a lo más dos
      procesos). Considera dos complejos de entradas, $\mathcal{I}$ y
      $\mathcal{I}'$, ambos consisten en dos simplejos $\sigma_{1}^2$,
      $\sigma_{2}^2$ de dimensión 2 (y todas sus caras). El primero
      los dos simplejos que comparte una arista, y en el segundo los
      dos simplejos que comparten un solo vértice. El ejercicio es
      describir el complejo de protocolo iniciado con $\mathcal{I}$, y
      también iniciado con $\mathcal{I}'$, para dos rondas. Describir
      en detalle estos complejos, y la vista de cada proceso en cada
      vértice, explicando intuitivamente el significado de los
      simplejos. Presenta estos complejos tanto en el caso coloreado,
      como en el no coloreado.
    } 
  }

\item{
    \textsl{
      Leer la entrevista y biográfica de Leslie Lamport
      http://www.technologyreview. com/news/525621/three-questions-for-leslie-lamport-winner-of-computings-top-prize/
      \\ http://research.microsoft.com/en-us/news/features/lamport-031814.aspx 
    }

    Leslie Lamport.\\
    Lamport, es un investigador principal en Microsoft Research, que acaba de
    recibir el premio Alan Turing de la ACM por su trabajo fundamental en teoría
    de computación distrubuida.\\
    Su artículo, de 1978, ''Time, Clocks, and the Ordering of Events in a
    Distributed System'' es uno de los más citados en las Ciencias de la computación.\\
    Ante el anuncio de la ACM, algunos personajes importantes como Bill 
    Gates, Butler Lampson, Harry Shum, Nancy Lynch, Ed Lazowska y Bob Taylor
    reconocieron el trabajo de Lamport.
    A pesar de los comentarios hechos, Lamport dice: ''No fui tan bueno encontrando
    soluciones. Pero si fui muy bueno para hacer preguntas''.\\\\
    El artículo principal de Lamport, ''Time, Clocks'', fue resultado de la
    introducción que escribió para un artículo llamado ''The Maintenance of Duplicate
    Databases'', en el argumentaba que se requería el uso de timestamps cuando dos 
    bases de datos identicas debían preservarse después un cambio hecho por una de
    ellas.\\
    Con esto Lamport ubico el problema de identificar el orden temporal de dos eventos 
    sin una conexión causal, lo cual produjo que entendiera que los timestamps podrían
    ser generador para obtener el orden de los eventos. Después dedujo que podría
    generar esto a partir de una máquina de estados.\\\\
    Otro de sus trabajos importantes es el ''Algoritmo de la panadería'', en el que
    propone una solución al problema de exclusión mutua en el que se corrompen datos
    dado que muchos procesos escriben en la misma memoria o que un proceso este leyendo
    antes de que otro proceso no haya terminado de escribir.\\
    El nombre del algoritmo viene de cuando en una panadería los clientes toman un
    boleto numerado cuando entran.\\\\
    En 1972, Lamport trabajo en un sistema de aviación de la NASA tolerante a fallos.
    Lamport observó que un fallo, podría ser que un proceso se detenga, o bien un
    proceso que comete equivocaciones o manda mensajes incorrectos. A estas fallas
    les llamo Fallas Bizantinas.\\
    La técnica mas usada es usar tres computadoras separadas para votar, en una
    regla de la mayoría, Lamport quería demostrar que esto funcionaba y entonces
    encontró el problema: necesitaba cuatro computadoras para evitar un fallo.\\\\
    Lamport también definió dos conceptos: ''Seguridad'' que es que nada malo pase
    y ''Vida'' que es que algo bueno pase.\\
    Para asegurar estos dos conceptos en un programa, Lamport se dio cuenta que
    no podía representar un programa con lógica temporal. Pero se le ocurrió en lugar
    de añadir operadores temporales cambiar las formulas lógicas de tal manera que en
    lugar de tener un estado, manejaba un par de estados (el actual y el siguiente).\\
    Con esto pudo crear TLA, una especificación de lenguaje para describir programas
    y sistemas.
    Después de que DEC, fuera comprada por Compaq, Lamport se integro a Microsoft,
    donde sigue trabajando.\\\\
    En una entrevista hecha a Lamport, donde se le pregunta ''¿Porque es tan
    importante la computación distribuida?'', ''¿Porque tus algoritmos aún
    sobreviven?'' y ''¿Que hay de malo en cómo se hace software ahora?'', Leslie
    destaca que en la computación distribuida no solo se trata de distribuir tareas
    si no de mantener las cosas coherentes, que su algoritmo Paxos se sigue usando 
    en Google y en Amazon porque los problemas y nociones de sincronización son los
    mismos de cuando los desarrolló y que la gente piensa que programar es solo hace
    código; dice que antes tienes que entender lo que haces, y lo relaciona con alguien
    que construye un puente sin un plano.\\
    Concluye que la mejor manera de hacer las cosas es con matemáticas.
  }

\item{
    \textsl{
      Leer y hacer un resumen de no más de 2 páginas, del articulo de
      Lesli Lamport, \textit{Time, clocks and the Ordering of Events
        in a Distributed System}. \\
      http://research.microsoft.com/en-us/um/people/lamport/pubs/time-clocks. pdf
    }
    
    El concepto ''un evento sucede antes de otro evento'' en sistemas
    distribuidos sirve para definir un orden parcial sobre el conjunto
    de los eventos de cualquier ejecución sobre un sistema
    distribuido. En el artículo escrito por Leslie Lamport se expone un
    algoritmo distribuido para sincronizar un sistema lógico de
    relojes que puedan dar un orden total al conjunto de eventos.
    Y finalizamos con un refinamiento del algoritmo para sincronizar
    relojes físicos.\\
    \\
    Asumimos que un sistema dsitribuido esta compuesto por una
    colección de procesos que se comunican a través del paso del
    mensajes, cada proceso consiste en una secuencia de eventos, en
    otras palabras, un proceso se define por un conjunto de eventos
    que tienen un orden a priori, en éste artículo no se discutirá el
    caso en donde cada proceso se puede dividir en subprocesos, hay
    que tomar en cuenta que recibir o enviar un mensaje hacia otro
    proceso también cuenta como un evento. Definimos la relación ''un
    evento sucede antes de otro evento'', denotado como
    ''$\shortrightarrow$''.\\ \textit{Definición:} La relación
    ''$\shortrightarrow$'' sobre un conjunto de eventos satisface las
    siguientes condiciones: 
    \begin{enumerate}
      \item{Si $a$ y $b$ son dos eventos en el mismo proceso, y $a$ se
        ejecute antes que $b$, entonces $a \shortrightarrow b$.}
      \item {Si $a$ es el envio de un mensaje en un proceso y $b$ es
          la recepción del mensaje por otro proceso, entonces $a
          \shortrightarrow b$.}
        \item{Si $a \shortrightarrow b$ $b \shortrightarrow c$,
            entonces $a \shortrightarrow c$.} 
    \end{enumerate}
    Decimos que dos eventos son concurrentes si $a \nrightarrow b$ y
    $b \nrightarrow a$.\\
    Con la definición anterior podemos definir un a serie de relojes
    abstractos. Sea $C_i$ el reloj del proceso $P_i$, tal que $C_i$
    asigna números a los eventos del proceso $P_i$ escrito como $C_i\langle a
    \rangle$, todos los relojes $C_i$ de cada proceso estan
    representados por una función $C$, tal que $C\langle b \rangle$
    $=$  $C_j\langle b \rangle$  donde $b$ es un evento del proceso
    $P_j$.\\
    Por el momento estudiaremos los relojes abstractos que hemos
    descrito, pero debemos hacer notar que estos no son relojes
    físicos.\\
    El proposito de este reloj abstracto, es el de proveer al sistema
    de un orden total respecto a que eventos ocurren antes que otros,
    por lo que decimos que la siguiente condición es suficiente.\\
    $Condicion:$ Para cada evento $a$, $b$ si $a\shortrightarrow b$
    entonces $C \langle a \rangle$ $<$ $C\langle b \rangle$.\\
    Para que la condición se cumpla dentro de nuestro sistema
    distribuido se debe satisfacer lo siguiente:

    \begin{enumerate}

    \item{$C1$ Si $a$ y $b$ son eventos del proceso $P_i$, y $a$ occure
          antes que $b$, entonces $C_i \langle a \rangle$ $<$ $C_i
          \langle a \rangle$.}

        \item{$C2$ Si $a$ es un envio de mensaje hecho por el proceso
            $P_i$ y $b$ es el evento del proceso $P_j$ que recibe ese
            mensaje, entonces $C_i \langle a \rangle$ $<$ $C_j \langle a
            \rangle$.}

    \end{enumerate}
    
    Para tener una primera aproximación de este reloj abstracto,
    podemos pensar que cada proceso tiene un contador, por el momento
    supondremos que  el contador de cada proceso $P_i$ esta
    sincronizado con los demás contadores.\\
    De esta forma para que esta pequeña aproximación cumpla las
    condiciones $C1$ y $C2$, debe suceder lo siguiente:

    \begin{enumerate}
      
      \item{$IR1$ Cada proceso $P_i$ incrementa su reloj $C_i$ entre
          cada dos eventos sucesivos.}

        \item{$IR2$ 
            \begin{enumerate}

              \item{Si el evento $a$ en el proceso $P_i$ envia un
                  mensaje $m$, entonces el mensaje $m$ contiene un
                  $timestamp$ $T_m$ $=$  $C_i \langle a \rangle$}

              \item{El evento $b$ del proceso $P_j$ que recibe el
                  mensaje $m$ tiene un $timestamp$ mayor a $T_m$.}

            \end{enumerate} 
          }
        \end{enumerate}
        
        Podemos usar este sistema de relojes que hemos descrito
        para describir un orden total para los eventos ocurridos
        en un sistema distribuido.\\
        Definimos la relación $\Rightarrow$ entre dos eventos, si
        $a$ es un evento del proceso $P_i$ y $b$ es un evento del
        proceso $P_j$, entonces $a \Rightarrow b$ si y solo si $(i)$
        $C_i \langle a \rangle$ $<$ $C_j \langle b \rangle$ o $(ii)$ 
        $C_i \langle a \rangle$ $=$ $C_j \langle b \rangle$ y $P_i
        < P_j$.\\
        Es importante notar que este orden depente del sistema de
        relojes del cual haga uso el sistema y por lo mismo el orden
        no es único.\\
        \\
        La capacidad de poder ordenar de forma total los eventos nos
        brinda la capacidad de poder resolver ciertos
        problemas. Recordemos  el problema de los filosofos propusto
        por Dijkstra,  tenemos $n$ procesos que comparte un solo
        recurso del sistema.\\
        Decimos que el siguiente algoritmo resuelve este problema.\\

        \begin{enumerate}
        \item{Para pedir el recurso, el proceso $P_i$ envia un mensaje
            $T_m:$ $P_i$ $solicita$ $el$ $recurso$ a todos los
            procesos y coloca el mensaje en su propia cola de solicitudes.
          }

        \item{Cuando $P_j$ recibe el mensaje  $T_m:$ $P_i$ $solicita$
            $el$ $recurso$, coloca el mensaje en su cola de
            solicitudes y envia un mensaje de recibido a $P_i$.}

        \item{Cuando el recurso es liberado, el proceso $P_i$ remueve
            cualquier mensaje $T_m:$ $P_i$ $solicita$
            $el$ $recurso$ de su cola y envia un mensaje $P_i$ $libera$
            $el$ $recurso$ a cada proceso.}

          \item{Cuando un proceso $P_j$ recibe el mensaje $P_i$ $libera
              el recurso$, remueve los mensajes $T_m:$ $P_i$ $solicita$
              $el$ $recurso$ de su cola.}

          \item{El proceso $P_i$ hace uso del recurso cuando alguna de
            las siguientes condiciones se cumple.

            \begin{enumerate}
              
            \item {Existe un mensaje  $T_m:$ $P_i$ $solicita$
                $el$ $recurso$ en su propia cola de prioridades que es
                el mensaje con más prioridad.
              }

              \item{$P_i$ recibe un mensaje de otro proceso con un
                  $timestamp$ menos antiguo a $T_m$. }
              
            \end{enumerate}

          }
        \end{enumerate}
        
 
      }

    \item{
      \textsl{
        Leer y hacer un resumen de no más de 2 páginas del capítulo de
        Leslie Lamport del Libro Out of Their Minds. \\ 
      }
    }

\end{enumerate}
\end{document}